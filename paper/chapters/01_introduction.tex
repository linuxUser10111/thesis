\chapter{Introduction}\label{chapter:introduction}

\ac{ML} is about the development of computer systems that can learn and adapt with following explicit instructions or any humany interference. And it uses special algoriths and statistical models to analyze and predict the outcome from a given pattern of data. It was born to allow computer to learn and control their environment.\newline
In today’s rapidly evolving landscape of machine learning and software engineering, the integration of \ac{ML} operations (\ac{MLOps}) plays a crucial role. \ac{MLOps} serves as the critical bridge between data science and DevOps practices, ensuring seamless collaboration and efficient deployment of \ac{ML} models. As organizations increasingly rely on \ac{ML} for decision-making, it becomes imperative to address the challenges associated with transitioning from local development to production environments.
\newline
To address these  challenges the goal of this Thesis is to try to bridge the gap between local development and successful code execution on the cloud, in other words, how to make \ac{ML} preocesses automated and operationalized so that more \ac{ML} proof of concept can be brought into production.
\newline
To overcome those chalanges, I conduct a mixed-method research
endeavor to identify important principles of \ac{MLOps}, carve
out functional core components, highlight the roles necessary to
successfully implement \ac{MLOps}, and derive a general
architecture for \ac{ML} systems design. In combination, these insights
result in a definition of \ac{MLOps}, which contributes to a common
understanding of the term and related concepts. 
\newline
The reminder of this thesis is constructed as follows.I will first
elaborate on the necessary foundations and related work in the field.
Next, i will give an overview of the utilized methodology,
consisting of a literature review, a tool review, and an interview
study. I then present the insights derived from the application of
the methodology and conceptualize these by providing a unifying
definition. I conclude the paper with a short summary,
limitations, and outlook.