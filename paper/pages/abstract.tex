\chapter{\abstractname}

%TODO: rewrite this part
\ac{AI} is a branch of computer science that aims to create systems capable of performing tasks that would typically require human intelligence. These tasks include learning and adapting to new information, understanding human language, recognizing patterns, solving problems, and making decisions. \ac{AI} can be categorized into two main types: narrow or weak \ac{AI}, which is designed to perform a specific task, and general or strong \ac{AI}, which can perform any intellectual task that a human being can do. AI has various applications, such as self-driving cars, virtual personal assistants, and image recognition systems.
\newline
\ac{ML} is a subcategory of \ac{AI}. It is based on algorithms trained for decisions making that automatically learn and recognize patterns from data. 
\newline
\ac{MLOps}, is a practice that combines machine learning, data engineering, and DevOps practices to streamline the deployment and management of machine learning models. \ac{MLOps} aims to bridge the gap between data science and DevOps, ensuring seamless collaboration and efficient deployment of \ac{ML} models. It involves various components, such as model versioning, data pipeline management, model monitoring and observability, testing strategies for models, deployment orchestration, and choosing the right tools. Effective collaboration between data scientists and DevOps engineers is essential for successful \ac{MLOps} practices.
\newline
\newline
This Bachelor thesis aims to highlight the difficulties of transitioning from local development environment to cloud based code execution in production environment in machine leaarning operations, and tries to find solutions to closeing the gap between local development and successful code execution on the cloud.